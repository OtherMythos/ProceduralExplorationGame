\documentclass[a4paper]{scrreprt}

%% Language and font encodings
\usepackage[english]{babel}
\usepackage[utf8]{inputenc}
\usepackage[T1]{fontenc}

%% Sets page size and margins
\usepackage[a4paper,top=3cm,bottom=2cm,left=3cm,right=3cm,marginparwidth=1.75cm]{geometry}

%% Useful packages
\usepackage{amsmath}
\usepackage{graphicx}
\usepackage[colorinlistoftodos]{todonotes}
\usepackage[colorlinks=true, allcolors=blue]{hyperref}

\newcommand{\nameofgame}{Voxel RPG game}

\title{\nameofgame{} - Game Design Document}
\subtitle{An RPG exploration game that fits into five minute bursts}
\author{OtherMythos}
\titlehead{\centering\includegraphics[width=6cm]{logo}}

\date{\parbox{\linewidth}{\centering%
  \color{red}
  V0.1.0 - Work in progress\endgraf
  \input{../.git/refs/heads/master}
  }}


\begin{document}
\maketitle
\newpage

\begin{abstract}
Explore a procedurally generated world collecting items, fighting monsters and discovering places.
Intends to distil the feeling of exploring a large fantasy world into a tight gameplay loop riddled with dopamine and quick flashy content.
\end{abstract}

\tableofcontents

% ______________________
% chapter Overview
% ______________________
\chapter{Overview}

% Main features and aspects of your game on a first page, describing story elements. -> "selling page", publisher should be able to decide after reading this single page whether to buy in or not

\nameofgame{} is a game generally targeted at mobile play styles, intended to distil the feeling of exploring a large fantasy world into something that can be easily picked up and played in short bursts.
It focuses on traditional RPG tropes to build its world and setting, using a voxel art style to keep its look simple.

\vspace{5mm}
Three words to describe the gameplay:

\begin{itemize}
    \item Quick
    \item Flashy
    \item Lots
\end{itemize}

\section{Quick}
Gameplay should be easily accessible and fit into short play sessions.
Animations, gameplay and interactions should fit into a short amount of time to keep the player engaged.
During gameplay something should happen every five seconds minimum.
The player is expected to make snap decisions regarding their health, stats, collected items and others.

\section{Flashy}
Animations should be fast and eye-catching.
The interface should emulate the feeling of using a slot machine, it should be well animated and a source of dopamine.
When the player achieves something important the interface should respond accordingly and provide congratulatory feedback.
The art style should be fun and engaging.

\section{Lots}
There should be lots of variety in the items, monsters and things to discover.
New items are gradually introduced to the player as the game plays out.
A variety of reward systems should make the game appealing and there should be lots of ways to utilise these items in ways that effect the gameplay.


% ______________________
% chapter Specification and Market Analysis
% ______________________

%\chapter{Specification}
%description of target group, platform, art style, who to attract of how to attract
%
%\section{Player(s) / Target-group}
%who is the target group?
%
%\section{Genre}
%what is the genre of the game?
%
%\section{Art Style}
%the art style of the game?
%
%\section{Forms of Engagement}
%thinking of Hunicke's 8 kinds of "fun" - what would you like to focus on?\\
%(1. Sensation - Game as sense-pleasure
%2. Fantasy - Game as make-believe
%3. Narrative - Game as drama
%4. Challenge - Game as obstacle course
%5. Fellowship -  Game as social framework
%6. Discovery - Game as uncharted territory
%7. Expression - Game as self-discovery
%8. Submission - Game as pastime)

% ______________________
% chapter Game Details
% ______________________


\chapter{Gameplay and Game Setting}
%be specific about the core game features

Gameplay follows the loop of:

    \begin{itemize}
        \item Begin exploration
        \item Find items, places, kill monsters and collect resources
        \item Reach the gateway and end the exploration
        \item Increase levels, trade items, progress the story
        \item Repeat
    \end{itemize}

The game is fundamentally about grinding and levelling.
After each exploration the player will be given options about how to progress their character, and the things they have collected on the previous exploration will be given to them.
Fundamentally the game involves moving around the world, so all these moment to moment gameplay systems must be fun.

\section{Exploration}
The main bulk of the gameplay takes place in the exploration scene.
The player is presented with a freshly generated world and must move around it, looking to find the gateway which allows them to successfully leave.
If the player runs out of health before they find the gateway, they will fail the exploration and lose the items they found during that exploration.
The player must manage their resources to avoid death and successfully improve their character.

\subsection{Input}
The player is placed into a procedurally generated voxel world and must move around to find the gateway.
The player is able to move the camera around them to help observe their surroundings.
In this sense, moving the camera to inspect the world should become a part of the micromanaging jobs that the player has to perform.
Depending on the input device the player can direct their character in a certain direction.

\subsection{Combat}
Combat follows a point and click approach.
The player can click on an enemy to target it and the system will initiate the combat.
Enemies which are targeting will walk towards each other and initiate the combat.
As well as this the player has a series of special moves which can be used once their recharge has finished.
Combat should be a system that the player has to micro-manage, rather than being too intrusive.

\subsection{Visiting Places}
During exploration gameplay the player has the option of visiting places.
This would include locations such as procedural dungeons, or smaller towns.
Places like dungeons give the player a small challenge while regular locations could provide a number of different things.
(Note I'm not so sure how this is going to work from a gameplay perspective.
It might be that the player can only visit towns when the exploration has finished.
I need to make sure it doesn't break the flow.)

\subsection{Terrain Types}
Different terrain types will help force the player to make choices about how they move their character around the world.
For instance, the player will move slower when moving through a body of water.
As the water gradually gets deeper the player will move slower as a result.
Similarly, climbing a gradient could slow the player down, allowing for some tactic to play into how to move around the world.

\subsection{Dropped Items}
Dropped items include exp orbs as well as money and inventory items.
The player simply has to walk towards them in order to collect them.
When the inventory is full, inventory items will remain on the ground, whereas money and EXP orbs can be collected in an unlimited amount.

\subsection{Biomes}
The exploration world is split into biomes to help break it up.
Biomes come in a number of forms and will contain different monster and found item types.
Biomes will integrate with the story and world building to help give context for their contents.

\subsection{Masked World}
As the player moves around the world, pieces of the world gradually become visible.
Invisible parts of the world help to make the player guess what sort of places will be in their randomly generated exploration.
This also adds an element of risk as the player has to decide whether to continue exploring or leave early once they've found the gateway.

As the player enters an area which was previously invisible it will become visible and a flashy animation should play.
Similarly, some hints as to what can be found in the invisible parts should exist, for example marking towns or settlements in the area with a glowing beacon.
This should help allure the player towards that location and the prospect of discovering something.

\subsection{Distractions}
The player should have some temptations trying to distract them while exploring.
This should help keep the player engaged and feeling on their toes.

Distractions would include things such as:

    \begin{itemize}
        \item Random encounters with NPCs
        \item 50/50 chance of receiving items if entering an area
        \item Items which trigger a chain of EXP orbs in a certain direction
        \item Chests in the overworld which can spawn items
    \end{itemize}

\subsection{Weather}
Weather events can occur during the exploration and have an effect on the gameplay.
For instance, rain might decrease the effect of certain combat moves, or increase the spawn rate of certain enemy types.
A lightning weather effect might make exploration in the world more precarious.
There should be a variety of weather patterns to help keep the gameplay interesting.

\subsection{Exploration Indicators}
The player should be offered a variety of items or events in the world that indicate what sort of things they can expect in their exploration.
For instance paths that link settlements to indicate life or items that indicate the direction of the gateway.
These things should help to guide the player and tempt them in a certain direction.

\subsection{Pre-Exploration Buffs}
Prior to the exploration beginning the player will be presented with a slot machine type minigame to select some exploration buffs.
These buffs would be modifiers such as 1.5x all money during exploration, 2x EXP orbs, guarantee a city will appear in the exploration, etc.
The primary purpose of this system is to hide the time it takes to generate a fresh world from the player.

At time of writing the world gen system takes roughly 5 seconds to complete (although this will be optimised).
As a result of this in order to provide a smooth gameplay experience it must be obfuscated with a small minigame which is actually relevant to the gameplay.
The system will be built in a way in which there should be no technical slowdown as a result of the world gen running.

\section{Exploration Locations}
Locations in exploration will provide some alternate variety as part of the exploration.
Within each location are biomes, which provide more granular variety to the exploration.
The purpose of locations is to help give the player progressively more of a challenge, as locations must be gradually unlocked through gameplay.

The current list of locations is:

    \begin{itemize}
        \item The Grasslands - Simple starter area
        \item The Desert - Minimal amounts of water. Sections of deep sand contain giant wurms.
        \item The Great Reservoir - The water levels rise and fall as the exploration progresses.
    \end{itemize}

It is important that each location contains a different gameplay theme or style, as they ultimately act like levels.

\section{Item Economics}
The game is about collecting items and progressing the character.
Each category of item must be well balanced and help facilitate the gameplay.

\vspace{5mm}
Items fit into the following categories:
    \begin{itemize}
        \item Equippables
        \item Consumables and generic items
        \item EXP orbs
        \item Money
        \item Charms
        \item Combat Moves - Maybe move this
    \end{itemize}

\subsection{Equippables}
Equippables are items such as swords, armour, accessories, etc.
The player has a few different equippable slots including head, body, legs and two accessory slots.
As well as this the player has two hands to equip weapons to.

Items give different stats depending on what they are and change the combat gameplay accordingly.
For instance two swords can be equipped at once to allow for a duel wield build.
Similarly the player should be able to make unconventional builds such as a character wielding two shields.
This variety of choice when building a character should help add variety to the game and allow more experienced players to challenge themselves.

\subsection{Consumables and generic items}
Consumables would be items such as potions which provide temporary benefits.
Similarly some items can be found which can be traded or sold, which is their main value.
The player is able to use these items mid way through the exploration, so there will be some tactic in their use.

\subsection{EXP orbs}
One of the main item systems are the EXP orbs.
These are collected during an exploration and can be used to level up the character in a variety of ways once exploration has finished.
They are one of the main systems to progress the character.

\subsection{Money}
Used to purchase items from a variety of locations such as shops.
Multiple currencies might be used as part of the world building to represent different kingdoms.
This could provide some gameplay variety as the player has to balance their money.

\subsection{Charms/Buffs}
TBD

\subsection{Combat Moves}
TBD

\section{Mood and Emotions}
The gameplay is intended to be fast and fun.
However, from a story perspective, the game takes itself quite seriously, although still maintains a cartoony feel.
The mood of the game is intended to follow a traditional RPG style and maintains a number of tropes, although does introduce some new concepts.
The game should include a heavy emphasis on lore, where each item, location or biome has some sort of detail in the lore.
This is intended to give more devoted players something to learn about.

\section{Story}
TBD

%\section{World/Environment}
%what is the settings of the game
%
%also, add here a map of your environment or a picture of your world if necessary
%
%\section{Objects in the Game}
%what objects will be in the game?
%
%\section{Characters in the Game}
%who are the characters in the game?
%
%\section{Main Objective}
%what is the goal / main objective of the game?
%
%\section{Core Mechanics}
%very important section: what are the core mechanics? be specific
%
%\section{Controls}
%describe the controls of the game
%also, add here a controller diagram if necessary

% ______________________
% chapter Front End
% ______________________


\chapter{Front End}
The user interface should be as simple and efficient as possible.
Given that this is a mobile first game, the interface must not impede the player in any way.
As well as being animated in an entertaining way (As mentioned in the 'flashy' objective), the interface must also be easy to understand and simple to use.

\section{Start Screen}
Ideally the title screen should be merged into the save selection screen in order to limit the number of screens between game start and gameplay.
The screen should present the title of the game as well as options to alter the save.
Otherwise the previous played save file should be used.
If the player wishes to continue with their previous save they should just be able to press a distinctly obvious button to begin exploration selection.

\section{Exploration Screen}
During exploration the user interface should be as minimal as possible, while also giving the player control over their character.
The exploration interface should primarily be taken up by the rendered scene.
Exploration effects such as coin or experience effects should also be present in the exploration screen to help sell gratification to the player.

% ______________________
% chapter Game Details
% ______________________


% ______________________
% chapter References
% ______________________

\chapter{References}
Games which have been studied include:

    \begin{itemize}
        \item Battleheart - Mika Mobile - 2011
        \item Cat Quest - The Gentlebros - 2017
        \item Cube World - Picroma - 2019
        \item Don't Starve - Klei Entertainment - 2013
    \end{itemize}

\section{Battleheart}
A favourite of the author's growing up, Battleheart condensed many RPG tropes into a fun and enjoyable mobile gaming experience.
Notable was its combat system which properly balanced micro-management and action into a control scheme which fit well onto a mobile device.

Fundamentally Battleheart is about progressing the player's party of characters through combat scenes.
At the end of each combat encounter the player will obtain loot which can be used to improve the party.
While it is a simple loop, the gameplay itself is engaging enough that it remains fresh hours in.

\section{Cat Quest}
An RPG game featuring a feline theme, Cat Quest was investigated as part of this project in relation to how it was ported to mobile.
The game was given a full mobile port, and attention was put into making the game work better on the mobile platform.
However from playing the game it became clear to the author that this game fundamentally was designed around traditional controller or keyboard input, rather than a touch screen.
Cat Quest as a game is significantly harder to play and a much less enjoyable experience on mobile when compared to another platform.

\section{Cube World}
A well known game in the indie space, Cube World is known for its procedural world and voxel art style.
Cube World leant into RPG tropes more than games such as Minecraft, as the game focused far more on experience and collectables.

\section{Don't Starve}
Don't Starve is a survival game based in a procedurally generated world.
The player must balance the needs of their character while also attempting to escape the situation they find themselves in.
Exploring the world is notable, for instance concepts such as different types of paths help guide the player in certain directions.

\chapter{Technology}
This game will be built using the avEngine, a game engine written entirely by the author of this document.
Part of the purpose of the creation of this game was as a way to test and develop the engine features and functions.

\section{Target Systems}
The game is intended to run on the three major desktop operating systems, Windows, MacOS and Linux.
As well as this there will be mobile ports for iOS and Android.
Devices such as the Steam Deck should be supported out of the box.

\section{Hardware}
Given the medium to low technical burden to run the game, it should be supported on most devices.

\section{Development Systems/Tools}
The voxel tool Goxel is being used to create models for this game.
A docker based pipeline has been created to translate these models into a format the engine can use.


\chapter{Marketing and Publishing Strategy}
Weekly videos are being made about the development of this game.
This game exists partly to grow the authors presence online as a game developer.
It is intended that a series of games will be created using the avEngine and published online with the intent of growing an audience.
This first game is intended to be a sizeable project that demonstrates the author's capabilities to produce scaleable projects rather than just small and low effort games.

\chapter{Timeline and Cost Estimation}

This game is intended to be a moderate sized indie game.
Much of the gameplay value should be derived from its replay value and random nature, meaning much of the effort to produce content can be simplified down to producing re-useable assets.

%\begin{table}[h]
%\centering
%\begin{tabular}{|l|l|l|}
%\hline
%Milestone & Description & Date \\\hline
%& Official Start Date & 01.12.... \\
%1 & Milestone Description ..  & 01.12.... \\
%2 & Milestone Description ..  & 01.01.... \\
%3 & Milestone Description ..  & 01.03.... \\
%& End of Project & 01.04.... \\
%\hline
%\end{tabular}
%\caption{\label{tab:schedule}Example Schedule.}
%\end{table}

\section{Time Estimation}

Initial development for this game went through a number of stages, notably the initial vision of a more idle adventure game was scrapped in favour of a more involved game.
This was partly as a result of a change in gameplay decisions but also in the direction of the author's online strategy.
It was decided that a larger and more involved game had to be created, instead deciding to spend upwards of a single year working on the re-thought game.

The estimate of time taken to complete this project relies on how much content eventually gets added to the game.
Depending on the initial success and traction the game receives, it is possible more or less content could be included.

This game is intended to be developed with a community focus.
The community will receive access to early builds of the game to help understand how the gameplay could be improved.
It is intended that new content will be drip-fed into community builds of the game after it has reached a certain point of completion.
This method will be useful to gather feedback from the community and help build up the gameplay.

\section{Intended Timeline}
The timeline for the time remaining in 2023 is as follows:

\subsection{Remainder of September}
Focus on distractions and guiding the player through the exploration.

    \begin{itemize}
        \item Make sections of the world invisible and make them appear when the player walks around.
        \begin{itemize}
            \item Add highlights for places which can be discovered as prospective things.
        \end{itemize}
        \item Improve how enemies appear
        \item Experiment with items which distract the player:
        \begin{itemize}
            \item Items which trigger a trail of EXP orbs
            \item Items which point out the direction of the gateway
            \item 50/50 encounters of a certain type
            \item Potential boss fights in exploration
        \end{itemize}
    \end{itemize}


\subsection{October}
Integrate a few of the remaining systems and flesh out the game.

    \begin{itemize}
        \item Add the charms and buffs system
        \item Implement the slot machine minigame before exploration begins
        \item Add some particle effects to help sell the combat encounters
        \item Begin trying to integrate more of the lore to make a complete product
    \end{itemize}

\subsection{November}
Technical and polishing work.

    \begin{itemize}
        \item Technical stabilisation - Make sure the Windows port functions, release builds should work.
        \item Polish the user interface up as much as possible, try and animate things and make it look as good as possible.
        \item Try and get ambient occlusion in place for regular models.
    \end{itemize}

\subsection{December}
Prepare product for initial release to community.

    \begin{itemize}
        \item Slack month, try and get builds of the game available for the start of the new year.
    \end{itemize}

\subsection{New year and 2024}
Focus on content, ideally with the gameplay and technical base focused and understood.

\section{Monetisation Strategy}
The monetisation strategy for this game on mobile will differ compared to the traditional Steam and PC release.
The PC release will include a flat fee when purchasing the initial game, with no extra potential cost to the player afterwards.

The mobile version will be free to download, but feature further monetisation strategies in game.
Emphasis has been put into making sure that the monetisation strategies used do not interfere with the gameplay experience in any detrimental way.

    \begin{itemize}
        \item The game will show adverts between each exploration session. - The player will be able to pay to disable them.
        \item The player will be given options to improve their pre-exploration buffs for cash.
        \item The player will be able to buy revives for their character after an exploration death for cash.
    \end{itemize}

Importantly, items which can be purchased with real money should always be able to be earned as part of regular gameplay.
The game should not be designed in a way which encourages players to purchase items with real money, they should only exist as an optional extra.

Given that part of the purpose of this project is to grow the author's online presence it is important that the playerbase feels satisfied with the finished product.
It is not a stretch to expect to receive monetisation from in game purchases, however this must be implemented in a way which is fair to the playerbase.

%\section{Cost Estimation}
%
%Estimated cost of the project based the described tasks and milestones and the  time estimation.

% ______________________
% chapter Game Details
% ______________________


\chapter{Team and Credits}

Credit for the idea, development and release strategy goes to Edward Herbert (OtherMythos)

Additional Credits:

    \begin{itemize}
        \item The YouTube viewer base
    \end{itemize}


%\todo[inline, color=green!40]{This is an inline comment.}

%\bibliographystyle{alpha}
%\bibliography{sample}

\end{document}
